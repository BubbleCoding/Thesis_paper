%This section talks about that which is retelling in video games
\section{Retelling games}

Game retelling is the act of reconstructing a past play experience into a new form whether through storytelling, video, illustration, or another medium. It is a way of making sense of what happened during play, translating fleeting moments into shareable narratives. In analog games like TTRPGs, where much of the story unfolds through improvisation and player interaction, retelling can serve many purposes. It preserves memories, communicates the experience to others, and allows for reflection on the themes, decisions, and dynamics that emerged. Whether casually recounted around a table, written into detailed session reports, or adapted into comics and podcasts, game retellings are always interpretive. They emphasize certain moments over others, shift the tone or point of view, and reshape nonlinear, messy play into something structured and communicable.

This process is not only about preserving content it is also an act of authorship. As soon as one begins to retell a game, choices are made about what matters, what is meaningful, and what should be retold. In doing so, the reteller inevitably reshapes the story, turning play into narrative. This transformation forms the foundation for deeper theoretical approaches to game retelling, including those that consider it a form of critique, reflection, or creative reinterpretation. This is double important when it comes to TTRPGs as these are games that are normally are not well recorded and live fully in the minds of the players with a couple spares notes on paper floating in the group.

Recapping is also a form of retelling. Many ongoing TV shows, series, anime, comic books, etc. tend to start each episode with a retelling of the previous episodes. Focusing only on the important aspects and leaving out a lot of the scenes. This is done in an attempt to reengage the watcher/reader and to remind them of the flow of the story so that they understand the story better. The same counts for TTRPG group who are part of a larger story that lasts numerous sessions separated by time. This is done in an attempt to restructure the story and make it clear to all the players to understand where they are in the story.

Retelling a game session, especially a TTRPG like Dungeons and Dragons, is not merely an act of documentation. It is a creative, interpretative, and often critical practice. In transforming a game  play experience into a new form, such as a comic book, the storyteller engages in a process of selection, transformation, and framing. This process raises questions not only about how stories are told, but also about what aspects of play are preserved, emphasized, or reimagined. After all a big part of the identity of a TTRPG is its rules and game play structures. Keeping play and the experience of playing in a retelling of a TTRPG is a key factor of conveying the story and the player experience.

Eladhari introduces the concept of game retelling as the fourth layer of narrative. According to Eladhari, this layer exists beyond the embedded, emergent, and interpreted narratives that typically characterize game storytelling. The fourth layer is concerned with post-play reflection and reinterpretation. It allows players, designers, or critics to re-engage with a game experience after the fact, often through a different medium (Eladhari 2018). In this view, retelling is not just about conveying what happened in the game, it is also about creating space for analysis, critique, or emotional processing.

Sych expands on this concept by arguing that critical game retellings can serve as a bridge between personal experience and broader thematic exploration. By combining Eladhari’s fourth layer with the idea of “breaking the fourth wall” Sych shows how retellings can expose the mechanics, social dynamics, and ethical tensions of gameplay itself (Sych 2022). This is particularly relevant in TTRPGs, where storytelling and rules are deeply intertwined, and where players often make decisions that reflect not just in-game motivations, but also real-world values and group norms.

Moreover, game retelling in this form invites consideration of what is lost and what is gained in translation. Elements like humor, tension, or spontaneity may be flattened or reshaped, while the visual format offers new tools, framing, symbolism, visual rhythm, to convey meaning. As Eladhari notes, retelling is not just representational it is generative. It produces new readings, new interpretations, and sometimes even new stories.

Finally, comic-based retelling also offers a lens into how different types of player interaction, such as roleplay, exposition, and rule engagement, manifest in a visual medium. The project thus becomes a way to explore not only how games are retold, but also how different aspects of a game are retold by artificial intelligence.


% Re-Tellings: The Fourth Layer of Narrative as an Instrument for Critique - Mirjam Palosaari Eladhari
% When the Fourth Layer Meets the Fourth Wall: The Case for Critical Game Retellings - Steven Sych

%This section showcases examples of retelling in video games
\subsection{Examples of TTRPG retellings}
%Adventure Zone (Comic) & Critical Role (Cartoon)

%This section showcases examples 1of AI based retelling/storytelling
\subsection{Artificial Intelligence as a Reteller/storyteller}
%AI Comic factory and The Finals commentator

%In this section we will look at comic studies and use that as a basis for the project
\section{Comics as a Medium for Retelling}
%Understanding Comics (Scott McCloud) and Comics and Sequential Art (Will Eisner)	
Creating Comic Books is an art form on its own and to understand how to generate a comic book with artificial intelligence, we need to understand how comic books function as an artefact. How they are made, how they are structured, how they are read, and what comic books are. This section will look at comic book studies to create a better understanding of comic books as an artefact.

\subsubsection{Comic Book a Defenition}
Just like how Scott McCloud starts his book \textit{Understanding Comics} we should also ask ourselves: What even is a comic book? Luckily we can just take the definition that McCloud gives us: "A comic book is a juxtaposed pictorial and other images in deliberate sequence, intended to convey information and/or to produce an aesthetic response in the viewer." \cite{mccloud1993understanding}. This defenition is very complex and for the purpose of this thesis we can simplify it to sequential art \cite{eisner2008comics}. When you take a comic book panel and look at it outside of the context of other panels it is just an image. But when you place them side by side in a sequential order the pictures become a comic book \ref{fig:SequentialArt}. Looking at the image we can see that an image of a man holding its hat is just a man holding a hat. But when you place two images of a man holding his hat side by side we can see that it creates a motion of tipping his hat. The panels create an effect of change, an effect of time passing. The space between the panels, knows as the "gutter", create a concept of time in stationary images. Placing iamges in a sequential manner is the essence of comic books.

\begin{figure}[!htp]
	\centering
	\includegraphics[scale=0.45]{images/SequentalArtUnderstandingComicPage5.png}
	\caption[Sequential Art]{Sequential Art page 5 \cite{mccloud1993understanding}}
	\label{fig:SequentialArt}
\end{figure}


\subsubsection{Comic Book Formats and Reading Conventions}

Comic books exist in a wide variety of formats and traditions, each with its own structural and cultural conventions. When people think of comic books, they often picture brightly colored superhero stories or comedic adventures they read as children. While these are certainly part of the medium's history, they represent only one branch of a much broader spectrum of comic book formats.

One of the most widely known formats is Japanese \textit{manga}. Manga is typically published in black and white, uses a right-to-left reading order, and often follows a clear panel structure with high variation in panel size and pacing. Manga tends to place a strong emphasis on emotion, movement, and stylized facial expressions, and often incorporates manpu—visual symbols to represent feelings or actions.

American comics, on the other hand, are usually published in full color and follow a left-to-right reading order. They tend to use grid-based layouts, with clearly delineated panels and speech bubbles. While the most globally recognized genres in American comics are superhero stories, the format is used for everything from noir and horror to memoir and historical fiction.

Korean \textit{manhwa}, and especially digital webtoons, represent another major tradition. These are primarily designed for vertical scrolling on digital devices, often in full color. Unlike page-based comics, webtoons use elongated vertical panels and spacing to control pacing and emotional beats. This format has become increasingly popular globally, thanks to the easy accessibility of webcomics on the internet.

Beyond these formats, there are countless experimental and hybrid approaches. European \textit{bande dessinée} , Chinese \textit{lianhuanhua}, and graphic novels each bring their own conventions. Some comics, such as Frank Miller's Sin City, use high-contrast black-and-white art to create a noir aesthetic \cite{miller1991sincity}. Others, like those illustrated by David Finch \cite{finch2024portfolio}, are known for hyper-detailed linework and realism.

These different comic book formats are not just stylistic choices—they influence how stories are told, how information is revealed, and how readers interpret visual cues. For any system that seeks to generate comics, understanding these formats is crucial. The chosen structure will affect panel composition, pacing, and even the kind of narrative that can be effectively conveyed.

\subsubsection{Visual Language and Narrative Techniques}
Pensl art style: Some comics even break the whole concept of panels and use whole pages as panels, place panels in panels, have panels that are not rectangular, but crooked, round, bend or anything else (Sandman). Some do not even use panels at all but make use of emptiness or use the same character on a long panel with multiple instances of the same character creating a concept of time as you read left to right. The important take away here is that comic books are not just one style, but many different styles.

Iconography is essential to many comic books to communicate those things that can not be easily spoken or written down. In the world of manga we call this unspoken communication \textit{Manpu} \ref{fig:Iconography}. To quote Matt Alt \textit{ Manpu "speak" what can't be easily spoken (or written) in words} \cite{GigaTown}. "ZZZ" to indicate someone is snoring or sleeping, a circle with starts and birds above a characters head indicate a state of unconsciousness by an impact of sorts, pages being torn of a calendar indicate time passing and one of my favourite making an L with your thumb and index finger and placing it under you chin to indicate deep thought (or the one I do not like where it represent people acting cool). 

\begin{figure}[!htp]
	\centering
	\includegraphics[scale=0.45]{images/IconographyGigaTown.png}
	\caption[Iconography]{Iconography \cite{GigaTown}}
	\label{fig:Iconography}
\end{figure}



\section{Generative AI Models for Multimodal Output}
There are many different AI models floating around the internet that could be used as part of the artefact that was made for this thesis. To choice which model to use this chapter talks about different models and why they were chosen for this thesis. In the end Mistral has been chosen as the large language model, FLUX.1 has been chosen for the image generation and Whisper has been chosen for the audio transcription model.

The requirements where these models where chosen on were:
\begin{itemize}
	\item Open source
	\item Locally runnable
	\item Good enough performance
	\item Ethical considerations as disclosed in section \ref{AIethics}
\end{itemize}
\subsection{Large Language Models}

Picking a suitable Large Language Model for the project is a quite the impossible task as models keep changing rapidly and new models are released every day. There are many academic papers out there discussing models and their performance, usage and more \cite{2024arXiv240201687A,chang2024survey,hadi2023survey,zhao2023survey}. However all of these are already more than a year old and as models are changing so fast most of those papers are already outdated, however still very useful. As this paper is not an analysis of AI models but a practical thesis I have asked ChatGPT-4-turbo to give me a list of models with their pros and cons and asked it to make me a table for Latex with the information. The prompts that I used: \textit{I would like to have a table with a comparison of the top 10 large language models. Show the pro's and cons of the models. Add if they are open source and locally runnable.} Followed by: \textit{Can you convert that table to Latex code?}. The results can be seen in table \ref{tbl:ModelComparison}, the latex code had to be modified to fit the page.

Even here we can see that the models are out of date. Gemini 1.5 Pro is not the latest model as Gemini 2.5 is already released, and LLaMA 3.1 is not the latest model as LLaMA 4 is already released. However, the table does give a good overview of the models that are available and their pros and cons. Primarily the question if they are open source and can be run locally. I used this method not because I see it as a academic valuable method but because I want to use technology and see how useful it could be for academia. Looking at this small experiment we can see that it could be useful, however the data is very outdated and not very useful. It does point us in the correct direction and gives us a good overview of the models that are available.

\begin{table}[ht!]
\centering
\resizebox{\textwidth}{!}{%
\begin{tabular}{|l|l|p{5cm}|p{5cm}|c|c|}
\hline
\textbf{Model Name} & \textbf{Developer} & \textbf{Pros} & \textbf{Cons} & \textbf{Open Source} & \textbf{Locally Runnable} \\
\hline
GPT-4o & OpenAI & 
Multimodal (text, image, audio); fast and versatile & 
Occasional hallucinations; high compute needed & 
No & No \\
\hline
Claude 3.5 Sonnet & Anthropic & 
Safety-focused; handles long contexts; natural responses & 
Slower; closed API only & 
No & No \\
\hline
Gemini 1.5 Pro & Google DeepMind & 
Strong multimodal support; integrates Google ecosystem & 
Resource-heavy; limited open variants & 
No & No \\
\hline
LLaMA 3.1 & Meta AI & 
Open-source; good at translation and code & 
Requires technical expertise; energy-intensive & 
Yes & Yes \\
\hline
Mistral Large 2 & Mistral AI & 
Strong reasoning and code generation; long contexts & 
Commercial license; proprietary & 
No & Limited (license) \\
\hline
DeepSeek-R1 & DeepSeek & 
Multilingual support; efficient & 
Limited adoption; less documented & 
No & No \\
\hline
Mixtral & Mistral AI & 
Efficient Mixture of Experts; open-source & 
Shorter context; niche usage & 
Yes & Yes \\
\hline
Grok 3 & xAI & 
Good social media integration; open-source focus & 
Requires subscription; limited distribution & 
Yes & Possibly (limited) \\
\hline
Command R+ & Cohere & 
Up-to-date info retrieval; good conversational skills & 
Limited multimodal support & 
No & No \\
\hline
Phi-2 & Microsoft & 
Open-source; efficient size/performance & 
Limited multimodal support; some bias concerns & 
Yes & Yes \\
\hline
\end{tabular}
}
\captionof{table}{Comparison table of 10 LLM's by ChatGPT-4-turbo}\label{tbl:ModelComparison}
\end{table}\noindent

Looking at the options provided we can quickly cut out most of the models as they are not open source and there is no possibility of running them locally. The models that are left are LLaMA, Mixtral and Phi. Out of these 3 models the choice is easy as Mixtral is the only options. Even do the other two models are open source, they are still owned by large American corporations (Microsoft and Meta) which are from personal point of view out of the question to use (More on this in \ref{AIethics}). Mistral is an privately owned and French company.

\subsection{Image generation models} 

\cite{bie2024renaissance}

%Talk about Stable Diffusion, Llama 3.1-8B and Whisper and why I chose them. Also talk about other models and why I did not choose them.

%In this section I will talk about the ethical consideration around AI usage.
\section{AI Ethics in Creative Applications} \label{AIethics}
Hello
%The Ethics of AI in games. – Melhart et al.
% Building ethics into artificial intelligence. – Yu et al.
% Deep Learning for Coders with Fastai and Pytorch: AI Applications Without a PhD – Jeremy Howard & Sylvain Gugger
% The ethics of Artificial Intelligence – Nick Bostrom & Eliezer Yudkowsky
% Think about stepping on the toes of artists and writers.

\section{The art of AI prompting}
Writing AI prompts is not something you can properly do without thinking about it. Writing proper prompts is a lot like writing code where you need to iterate over the instructions you give the computer to do what you want it to do. It is also know as prompt engineering. There are a lot of nuances to writing prompts and hidden tricks and commands to instruct an AI to get an AI to do exactly what you want it to do. This section discusses the basics of AI prompting and how they have been used in the artefact.

\cite{dang2022prompt}
%https://scholar.google.com/scholar?hl=nl&as_sdt=0%2C5&q=How+to+prompt+ai&btnG=

\section{Three primary forms of player interaction}
GPT:

Dungeons and Dragons sessions have three primary forms of player interaction: roleplay, exposition, and rule interaction. These three forms are distinctly different, and in this project they will be used to analyze different parts of the generated comic book to see if the system is better at certain parts.

Roleplay is when players and the dungeon master talk to each other while taking on the role of their characters. The dungeon master often juggles multiple roles, shifting between NPCs as needed. This kind of interaction mostly happens in-character and aligns with what game researchers call the narrativist mode from GNS theory, where the focus is on creating a story together (Edwards 2001). It also maps onto what is known in RPG studies as in-character talk, one of the core communication styles during play (Bowman 2010).

Exposition is when the dungeon master describes a situation, a location, or the mood of a scene. It is about creating a vivid image of what is happening in the game. This is not necessarily roleplaying, but it is key to setting the stage. It sits somewhere between narrativist and simulationist modes in GNS theory, and would be considered out-of-character talk, since it is often descriptive rather than spoken as a character (Bowman 2010).

Rule interaction is where the game mechanics kick in. This is when players roll dice and use other rules of the system to resolve conflicts, check if they hit an enemy, try to jump through a window, activate an ability, or make a persuasion check to convince someone that they are innocent. These moments are very gamist in nature. They are about winning challenges through the rules of the system. In terms of communication, this fits under meta talk, which includes discussing rules, tactics, or numbers, often from outside the fiction (Bowman 2010).

These three types of interaction not only shape how players experience the game, but also how that session is interpreted when turning it into a comic. Based on these three categories the final comic book result will be analyzed to see what is the difference to retelling these parts and to see what the differce is inbetween generation.


\section{Design Science Research as a Methodology}