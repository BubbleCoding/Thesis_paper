%This document is used to write text that does not have a proper location in the paper yet. I found that while writing I have a lot of issues understanding where to put certain text. This document is used to write that text and then move it to the right location in the paper.


\section{Design Science Research}

This thesis is based on the Design Science Research (DSR) methodology \cite{dresch2015design} and Computing as a discipline \cite{denning1989computing}. 

Computing as a discipline presents us with a framework for designing artefacts from an engineering point of view. It gives us four steps to solve a given problem:
\begin{enumerate}
    \item State requirements
    \item state specifications
    \item design and implement the system
    \item test the system
\end{enumerate}

For the requirements of this thesis we need to very specific as the thesis is working with two different requirements at the same time. The main research question is looking for the requirements of a comic book generating artefact thus the requirements of the research project should be linked to finding these artefact requirements. But that means that the requirements of the artefact itself can not be part of the requirements of the research project. The artefact requirements are part of the conclusion of the the thesis.


\subsection{Diverting from the DSR methodology}
In DSR you should start

---------------------------------------------------
\section{Retelling games}
Most people play games during their personal down time and do not really think about games outside of leisure context (REF). However there are also a lot of people that play games a lot and make it part of their lives. Looking at social media platforms like X, YouTube, Reddit, TikTok, etc. we can see that the sharing of gaming stories is a very big part of the internet. This sharing of stories can also be called retelling. Where people retell their personal stories about their gaming experiences. Retelling is not limited to games as the art of retelling is as old as humans and is a very big part of human culture(REF).

What is the essence of retelling? 

When talking about game retelling 


\subsection{Difference between retelling and storytelling}
\subsection{State of the art of retelling}
People share their gaming stories in many different ways. To have an holistic view of game retelling this section discussed the different forms of retelling video games.

Oldschool: talking
Podcast
YouTube
    Game play recording
        Podcasting
    Storytelling
    

An argument for Streaming.
For completion sake I want to mention streaming and unedited playthrough videos here as well. Streaming is a weird one when it comes to retelling as it is not a retelling of an experience, but a live experience.
    




\textit{Reteller is an agent that produces and narrates a sequence of events, for the benefit of either human players or spectators}\cite{Gallotta2024LLM}.

Retelling is one of the largest if not the largest cultural phenomenon out there. I would even say that the retelling is one of the reasons the internet has become as big as it is. All forms of social media became the behemoths they are because of people retelling their lives. Academics would not exist if we did not retell everything we do in the form of papers, books, lectures, etc. For this thesis- it is important to know what retelling is, how people interact with it, what the core aspects are of a retelling and which moments of the gameplay session are important and should be part of the retelling.